\documentclass[12pt,letterpaper]{report}
\usepackage[utf8]{inputenc}
\usepackage{amsmath}
\usepackage{amsfonts}
\usepackage{amssymb}
\author{Salvatore Testa}
\title{Hands-on with the Maker Movement}
\begin{document}
\begin{center}
  {\Huge \sc Hands-on with the \\Maker Movement}\\
  {with Salvatore Testa (sal.testa@rice.edu)}\\
  {\small Spring 2014 Syllabus}
\end{center}
\textbf{When: } 7-8:30PM Wednesday\\
\textbf{Where: }Abercrombie A119\\
\textbf{Office Hours:} by appointment\\
\textbf{Course Format and Description}\\
The goal of this course is to inform and excite students about the Maker Movement, the present day technological revolution emerging as a result of inexpensive yet powerful programmable electronics and low cost fabrication tools such as 3D printing. The curriculum will focus on in-class projects which will be supplemented by short lectures and class discussion. To help cover the cost of hardware and 3D printing, every student needs to bring \$30 to the first day of class.\\
\textbf{Course Requirements}\\ 
This course is designed for students with little to no programming and hardware experience. Every student should have their own laptop, but accommodations can be made for those who don't.\\
\textbf{Class Credit and Grade}\\
This is a one credit, satisfactory/unsatisfactory course. The in-class projects will make up 50\% of the grade, class participation (including attendance and discussion) will constitute another 30\%, and the final project will make up the remaining 20\%. To pass the class, you must have greater than a 60\% average and no more than 2 unexcused absences. If you are going to miss class, please let me know at least a day in advance, and we will reschedule a time for you to make up for the lab you missed.\\
\textbf{Rice University Disability Accommodation Policy}\\
Any student with a documented disability needing academic adjustments or accommodations is requested to speak with the course instructor during the first two weeks of class. All discussions will remain confidential. Students with disabilities should also contact Disability Support Services in Allen Center 111.\\
\textbf{Honor Code}\\ 
Students are expected to abide by the Rice Honor System (honor.rice.edu).
\section*{Course Outline}
\subsection*{1. Intro, History, and Setup}
Learn about the brief history of the maker movement, and make sure everyone's environment is setup properly. I will collect money for course materials at this time.
\subsection*{2. User Interfaces in Processing}
Create a basic, user friendly window with buttons that respond to input.
\subsection*{3. Hardware Outputs}
Overview of the output on the Arduino as seen through LEDs on a breadboard.
\subsection*{4. Soldering}
Learn soldering safety and make a small insect out of electronic components.
\subsection*{5. First Milestone}
Solder together a 3x3x3 LED grid, and make it light up.
\subsection*{6. 3D printing}
Discuss 3D printing and it's modern application as well as print out basic shapes.
\subsection*{7. Traffic Light}
Wire-up 3D printed traffic lights and have them blink in sequence.
\subsection*{8. Hardware Inputs}
Learn basic input using infrared sensors to detect when a toy car is near the traffic light and get the light to change colors.
\subsection*{9. Motors}
Control motors and learn to calibrate the motors with basic input.
\subsection*{10. Not-so-lazy Susan}
Create a platform that rotates with input from the laptop keyboard.
\subsection*{11. Begin Final Project}
Start on ping pong ball launcher, using components from the rotating platform
\subsection*{12. Continue Final Project}
Continue working on launcher, physical unit should be assembled by end of class.
\subsection*{13. Final Project Wrap-up}
Work on user interface to control the launcher.
\subsection*{14. Demo Final Project}
See who's got the best launcher through a series of challenges.
\end{document}